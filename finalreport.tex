\documentclass[11pt]{article}
\usepackage[T1]{fontenc}
\usepackage[english]{babel}
\usepackage{fullpage}
\usepackage{graphicx}
\newcommand{\HRule}{\rule{\linewidth}{0.5mm}}
%\def\wl{\par \vspace{\baselineskip}}

\begin{document}

% TITLE PAGE
\begin{titlepage}
	\begin{center}

% Header
\textsc{\LARGE AA241X: Design, Construction, and Testing of Autonomous Aircraft}\\
\HRule \\[0.4cm]
\includegraphics[width=1.0\textwidth]{./skynet}~\\
\HRule \\[0.4cm]
\textsc{\Large Final Report}\\
{\large \today}
\vfill

% Authors
\begin{minipage}{0.4\textwidth}
	\begin{flushleft} \large
	\emph{Authors:}\\[0.5cm]
	Kartikey Asthana\\
	\texttt{kasthana@stanford.edu}\\[0.5cm]
	Peter Blake\\
	\texttt{psblake@stanford.edu}\\[0.5cm]
	Brandon Jennings\\
	\texttt{bjennin@stanford.edu}\\[0.5cm]
	Erik Moon\\
	\texttt{emoon1@stanford.edu}\\[0.5cm]
	Sravya Nimmagadda\\
	\texttt{sravya@stanford.edu}\\[0.5cm]
	Akshay Subramaniam\\
	\texttt{akshays@stanford.edu}\\[0.5cm]
	Ian Villa\\
	\texttt{ianvilla@stanford.edu}\\[0.5cm]
	Jerry Watkins\\
	\texttt{watkins2@stanford.edu}\\[0.5cm]
\end{flushleft}
\end{minipage}
\begin{minipage}{0.4\textwidth}
	\begin{flushright} \large
	\emph{Degree \& Department:} \\[0.5cm]
	Ph.D. Candidate\\
	Aeronautics \& Astronautics\\[0.5cm]
	M.S. Candidate\\
	Graduate School of Business\\[0.5cm]
	M.S. Candidate\\
	Mechanical Engineering\\[0.5cm]
	M.S. Candidate\\
	Graduate School of Business\\[0.5cm]
	Ph.D. Candidate\\
	Aeronautics \& Astronautics\\[0.5cm]
	Ph.D. Candidate\\
	Aeronautics \& Astronautics\\[0.5cm]
	B.S. \& M.S. Candidate\\
	Aeronautics \& Astronautics\\[0.5cm]
	Ph.D. Candidate\\
	Aeronautics \& Astronautics\\[0.5cm]
\end{flushright}
\end{minipage}
\end{center}
\end{titlepage}

% TABLE OF CONTENTS
\clearpage
\tableofcontents
\vfill
\clearpage

\section{Introduction}
	\label{Introduction}
	Since the early 2000's, Stanford Aeronautics and Astronautics has taught the AA 241X: Design, Construction and Testing of Autonomous Aircraft course with various missions over the years. In Spring of 2014, teams were tasked with developing an autonomous aircraft to search and accurately locate four targets within the perimeter of Lake Lagunita. Among these teams was Skynet, a group of eight individuals from different backgrounds and expertise who, throughout the ten weeks, collaborated to organize, design, test, and fly various aircraft, guidance, control, and mission systems to optimally complete the search and rescue. The following report outlines the team's structure, mission strategy, aerodynamic design, control strategy, fabrication accounts, flight test data, and overall competition performance.

	\section{The Team}
		\label{Team}
		\subsection{Team Structure}
		\label{TeamStrc}
		Skynet began as a team of eight individuals from Stanford Aeronautics and Astronautics, Mechanical Engineering, and the Graduate School of Business. In order to effectively tackle the assigned problem sets and deadlines, the team began by designating Ian as team lead and the other members as sub team leaders in the different disciplines required by the class. These areas were Aerodynamics/Configuration Design/Propulsion, Guidance/Navigation/Control, Hardware/Fabrication, Software, Aircraft Performance, and Mission Planning as outlined in the first lecture by Professor Alonso. The initial team structure is outlined below.

		\begin{table}[!ht]
			\begin{center}
				\begin{tabular}{| p{7.5cm} | c | p{3.7cm} |}
					\hline
					\textbf{Initial Design Sub-Teams} & \textbf{Lead} & \textbf{Members} \\ \hline
					Aerodynamics/Configuration Design, Propellers/Propulsion & Jerry, Akshay, Kartikey & Sravya, Ian \\ \hline
					Guidance, Navigation \& Control & Brandon & Jerry, Sravya, Akshay \\ \hline
					Hardware & Erik & Brandon, Peter \\ \hline
					Software & Ian & Brandon, Erik, Jerry, Akshay, Sravya, Kartikey \\ \hline
					Aircraft Performance & Sravya & Jerry, Akshay, Ian, Peter, Kartikey \\ \hline
					Mission Planning & Peter & Everyone \\ \hline
					Fabrication & Erik & Brandon, Peter, Akshay, Jerry, Kartikey \\ \hline
				\end{tabular}
				\caption{Initial Team Divisions}
				\label{initTeam}
			\end{center}
		\end{table}


		\subsection{Team Communication \& Logistics}
		\label{TeamCommLog}
		In order to facilitate group discussions, a when2meet form was utilized online. Based on its results, the team met briefly after class on Mondays and Wednesdays for brief sub team status updates and coordination. Major team meetings were held on Fridays during the typical class time on the second floor of Durand and were spent discussing topics requiring everyone's attendance such as aircraft design and mission strategy.

		The team also utilized online methods to meet communication needs. A Ggioogle Group was utilized for formalized notices and e-mail discussions. Short-form and quick information relays were handled by a GroupMe that could be accessed via phone or computer.

		Data Storage and problem set completion was made possible via our Google Drive, Google Docs, and a Wordpress. All team data, code, and photos were uploaded into categorically defined folders in our Google Drive. Spreadsheets recording budget, weather data, contact information, useful links, and most importantly, problem set requirements were also held here. Having all of these documents in a single location and accessible by all of the team was the last step in facilitating good communication and ensured proper problem set completion. Once written, relevant text, data, graphs, and videos were uploaded to skynet241x.wordpress.com.

		\section{Mission}
			\label{Mission}
% KARTIKEY

\section{Vehicle}
	\label{Vehicle}
	\subsection{Design Approach}
	\label{DsignAppr}
	\subsection{Performance Characteristics}
	\label{PerfChar}
	\subsection{Flight Performance}
	\label{AeroFlightPerf}
% AKSHAY & SRAVYA
% Include in this section a Three View Drawing of the Plane


\section{Controls}
	\label{Controls}
	\subsection{Control Strategy}
	\label{CtrlStr}
	\subsection{Flight Performance}
	\label{CtrlFlightPerf}
	% BRANDON
% Include in this section Initial Approach vs. Final Choices


\section{Fabrication}
	\label{Fabrication}
	\subsection{Prototype Construction Approach}
	\label{ProtoConsAppr}
	\subsubsection{Mk-I "The Red Baron"}
	\label{mk1}
	\subsubsection{Mk-II "The Pig"}
	\label{mk2}
	\subsubsection{Mk-III.1}
	\label{mk3.1}
	\subsubsection{Mk-III.2 "Ronald McDonald"}
	\label{mk3.2}
	\subsubsection{Mk-III.3 "Terminator"}
	\label{mk3.3}
	\subsubsection{Mk-III.4 "The UltraLight"}
	\label{mk3.4}

	\section{Flight Testing}
		\label{FlightTesting}
		\subsection{Flight Test Approach}
		\label{FltTstAppr}
		\subsection{Simulation vs. Actual Tests}
		\label{simvsact}
% JERRY


\section{Mission Flight Results}
	\label{MissionFlightResults}
	\subsection{Official Flight Results}
	\label{OffFltRes}
	\begin{table}[!ht]
		\begin{center}
			\begin{tabular}{ | c | c | c | c | c | p{3cm} |}
				\hline
				\textbf{Flight} & \textbf{Date} & \textbf{Phase 1} & \textbf{Phase 2} & \textbf{Total} & \textbf{Comments} \\ \hline
				1 & 6/6/2014 &  &  & 76.84 & \\ \hline
				2 & 6/6/2014 &  &  & 50.26 & Dropped\\ \hline
				3 & 6/6/2014 &  &  & 25.27 & Dropped\\ \hline
				4 & 6/6/2014 &  &  & 74.61 & \\ \hline
				5 & 6/6/2014 &  &  & 35.76 & Dropped\\ \hline
				Overall & \multicolumn{5}{c|}{75.725}\\ \hline   
			\end{tabular}
			\caption{Official Flight Results}
			\label{flighttab}
		\end{center}
	\end{table}


	\subsection{Analysis of Flight Data}
	\label{AnalFltData}
% JERRY & KARTIKEY
% Include observations about about the validity of our approach.
% What observations from the data might inform future work (beyond the scope of the class).


\section{Conclusions \& Lessons Learned}
	\label{Conclusion}
% EVERYONE

\section{Future AA241X Recommendations}
	\label{Recommendations}
% EVERYONE

\end{document}